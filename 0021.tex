% Auto-slash processed
% \subsection{バイト先へ誘ってみる}
\subsection{别忘了把标题填到这里}

% 「私、ライトノベルに間違った偏見を持っていたわ」
「原来我对轻小说一直抱有错误的偏见耶」

% 「おっ、わかってくれた?」\\
「唷,是喔」\\

% また中庭でお昼を食べていると、小雪が改まった調子で頭を下げた。
中午两人又在院子里吃午饭时,小雪一本正经地低下了头。

% それに直哉は苦笑いを返すのだ。
对此直哉苦笑着回应。

% どうやら小雪はあの一冊で、ライトノベルに対する認識をすっかり改めてしまったらしい。\\
多半是由于那本书,小雪彻底改变了对轻小说的认识。\\

% 「笹原くんが『えっち』なものを読んでるって早合点しちゃったのも……本当にごめんなさいね。失礼なこと言っちゃったわ」
「而且我轻易下定论说你就是在看『色色』的东西……真是对不起了,说了些没礼貌的话」

% 「お、おう。いいってことよ」\\
「哦、喔。没关系啦」\\

% 直哉はぎこちない笑みを返してみせる。\\
直哉自然地回了一个笑脸。\\

% (いやまあ、貸したのは普通のやつだけど……当然『えっち』な漫画とかラノベも読んでるんだよなあ)\\
(哎呀那个啥,我借给她的是那种很普通的书……当然『色色』的漫画和轻小说之类的我也在看就是了)\\

% 小雪に勧められないような過激なラブコメも、当然所持している。男の子なのだから仕方ない。
直哉当然也有那种不推荐小雪看的过激爱情喜剧,毕竟是男孩子,这也是没办法的。

% だがしかし、それが好きな女の子にバレるのは絶対に阻止したかった。
然而,他绝对想阻止这种事情暴露给自己喜欢的女孩子知道。

% ちょっぴり冷や汗をかく直哉に気付くこともなく、小雪はほうっとため息をこぼしてみせる。\\
小雪舒了一口气,并没有注意到旁边的直哉略微擦了一把冷汗。\\

% 「ほんとに胸躍る冒険活劇だったわ……とくに最初は主人公にツンツンしていたフランちゃんが素直になるところとか、もう、ほんと……あれが死亡フラグってやつだったのね……」
「真是令人激动的冒险武斗耶……特别是一开始对主角盛气凌人的フランちゃん后来也变乖顺了,哎呀,真的是……那个是死亡flag对吧……」人名未替换

% 「あはは……どんまい」\\
「啊哈哈……别放在心上」\\

% しょんぼり沈んで顔を覆う小雪のことを、ネタバレをぐっと堪えて慰める。
直哉忍住不剧透,安慰着垂头丧气的小雪。

% 一巻で死んだ……と思われているクーデレキャラがどうやらお気に入りらしい。自分と似ているから、シンパシーを感じているのかもしれない。\\
第一卷就死了……貌似会被人这么以为的冷娇角色,她好像蛮中意的。说不定是因为很像她自己,所以感受到了共鸣吧。\\

% 「それじゃ二巻も読む? そう思って持ってきてるんだけど」
「那么第二卷也看看吧?我就是想让你看才带来的……」

% 「ありがたいけど、もう読んでる途中なのよね」\\
「谢谢你,不过我已经在看了呢」\\

% そう言って、小雪は二巻を取り出してみせる。\\
小雪这样说着,拿出了第二卷。\\

% 「妹が全巻持ってたの。やっぱり持つべきものは家族だわ」
「我妹妹有全套呢,果然家人是很重要的耶」

% 「へえ。朔夜ちゃん、けっこういい趣味してるなあ」
「欸,你妹妹朔夜ちゃん有个相当不错的兴趣啊」% 人名未替换

% 「え、妹のこと知ってるの?」
「诶,你认识我妹妹?」

% 「うん。昨日偶然出会ったんだ」\\
「嗯,昨天偶然碰到之后认识的」\\

% 実際には拉致監禁されたのだが。それは黙して語らず、さらっと流しておいた。
实际上直哉是被绑架监禁了,但是他缄口不提这事,而是轻快地一笔带过了。

% 小雪も深くツッコミを入れることなく、二巻の表紙を嬉しそうに撫でる。\\
小雪也没有深入吐槽,很开心地摸着第二卷的封面。\\

% 「えへへ。休憩時間にちょっとずつ読んでるの。だから放課後にお話ししましょ」
「诶嘿嘿,每到休息时间我都会看一点呢。放学后我们聊聊吧」

% 「ああ、もちろ……あっ、今日はダメだ。バイトだった」\\
「好,那当然可……啊,今天不行。我还有打工」\\

% そこで直哉ははたと気付く。
这时直哉一下子想起来。

% 今日は金曜日。バイトの日だ。そう伝えると小雪がへにゃりと眉を下げてみせる。\\
今天是周五,是打工的日子,他这样告诉小雪。小雪听完之后皱起眉梢,样子有些不愉快。\\

% 「あら、そうなの。それはちょっと残念ね……せっかくお話できると思ったのに」\\
「哎呀,是嘛。那可有点遗憾呢……我以为好不容易能聊聊了呢」\\

% 小雪が視線を落とす先は、例のライトノベルだ。
小雪垂下视线,看向那本轻小说。

% そのタイトルを見て……直哉はぽんっと手を叩く。\\
直哉也跟随她的目光……看到那个标题之后,直哉砰的一声拍了拍手。\\

% 「あ、でも逆にありかも」
「啊,但是这样说不定反而更好」

% 「へ?」
「欸?」

% 「白金さんが俺のバイト先に遊びに来たらいいんだよ。そしたら話もできるだろ、店長も喜ぶよ」
「你来我打工的地方玩就行了。这样就能聊天了,而且店长也会高兴的」

% 「ええ……お仕事の邪魔しちゃ悪いわよ」
「诶……打扰你干活不太好吧」

% 「別にかまわないって。店長が道楽でやってるような店だし」\\
「没事的。店长开那家店只是个兴趣爱好而已,没那么多规矩的」\\

% 直哉のバイト先は古本屋だ。滅多に客は来ないし、店の仕事といえばたまーにご近所のお得意様へ配達するくらい。
直哉打工的地方是旧书店。几乎没什么客人来,要说店里的工作,顶多也就是偶——尔给住在附近的老主顾送送书而已。

% 完全に儲け度外視で、店主がまっとうな社会生活を送るためだけに店を開けていると言ってもいい。
可以说,店主根本不考虑赚钱,只是为了在社会上有个正经职业才开的这家店。

% そう説明すると、小雪は首をかしげてみせるのだ。\\
直哉这样说明后,小雪歪了歪脑袋。\\

% 「道楽のお店でバイトを雇うって変な話ね……いったいどんな人なのよ、そこの店長さんって」
「自己的兴趣爱好却还招打工的人,真是怪事呢……那里的店长,到底是个怎样的人哪」

% 「そうだなあ、一言で言えば……」\\
「是啊,用一句话来说……」\\

% 直哉は慣れ親しんだ店長の顔を思い浮かべる。
直哉的脑海中浮现出了店长熟悉而亲切的面庞。

% 色々と属性過多な人ではあるものの、強いて言うなら――。\\
店长是个有着各种各样属性的人,不过硬要说的话——\\

% 「大人のお姉さん、って感じかなあ」
「成年大姐姐,这种感觉吧」

% 「…………へえ?」\\
「…………欸?」\\

% その言葉を発した瞬間、小雪の眉がぴくりと動いた。ついでにまとう空気がピリピリし始める。凛とした面持ちで小雪は静かに口を開いた。\\
发出这个声音的同时,小雪的眉毛抽动了一下。紧接着周围的气氛也开始紧张起来。小雪表情严肃而凛然,静静地开口说。\\

% 「わかったわ。行きましょう。むしろ絶対連れてって」
「我知道了,那么一起去吧。不如说一定要带我去看看」

% 「あ……いやでも、白金さんが思ってるような人じゃないから。俺と店長じゃフラグの立ちようがないっていうか、なんていうか」
「啊……不过,店长并不是你想象中的那样。怎么说呢,我跟店长没法立什么flag的……」

% 「結婚されてるとか、彼氏がいるとか?」
「因为她结婚了,或者有男朋友了?」

% 「いや、どっちもないけど」
「不,都不是……」

% 「じゃあ行く」
「那么我要去」

% 「はあ……俺は白金さん一筋なんだけどなあ」
「唉……我对你可是一心一意的啊」

% 「あなたはそうかもしれないけど、他の女の人がどう思うかわからないでしょ!!」\\
「你说不定可能是这样,但是别的女人就不好说了吧!!」\\

% 小雪はプンプン怒ってお弁当を食べ進める。
小雪怒气冲冲地开始吃起饭盒里的饭。

% 今のは素直になった結果か、追い詰められて口が滑ったのか。たぶん後者だ。\\
刚才她是老实交代,还是被逼急之后说漏嘴了呢,大概是后者吧。\\

% (うーん……だいぶ誤解されてるみたいだけど……ま、いいか。会ったらわかるだろ)\\
(唔……看来我被误解得很厉害啊……算了,就这样吧。见到之后就明白了吧)\\

% だから直哉もまた大人しく弁当を食べ進めるのだった。絶対にフラグが立つはずないと、会えばすぐに分かるはずだから。
所以直哉也老老实实地吃起饭盒里的饭来。这是因为直哉确信,小雪见到店长后马上就会明白,flag是绝对不可能立起来的。
