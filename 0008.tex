% Auto-slash processed
% \subsection{登校前からグイグイいく}
\subsection{从上学前就开始不断进攻}

% 次の日の朝。
第二天的早上。

% 学園の最寄駅の改札で待っていると、小雪が足早にやってきた。まだ待ち合わせには電車三本も早い時間だ。
在离学校最近的车站的检票口处等了一会儿后,小雪快步朝我走了过来。比约定的时间要早上那么两三趟电车的时间。

% そんな彼女に歩み寄り、直哉は片手を上げてみせる。\\
直哉朝她走去,举起一只手朝她打招呼。\\

% 「おはよー、白金さん。今日は俺が一番乗りだな」\\
「早上好,白金同学。今天我先到呢。」\\

% それに小雪は顔を赤らめて「お、おはよ……」と小声で返すのかと思いきや――。\\
然后脸颊变得绯红的小雪似乎低声地回答了一句「早,早上好……」\\

% 「ちっ……!」\\
「切……!」\\

% 実際に飛んできたのは威嚇のような舌打ちだった。
事实上传来的却是一声恐吓般的咋舌。

% 今のは照れ隠しなどではない。純度百パーセントの本気だ。
刚刚那并不是为了掩饰害羞什么的,是她百分百的真实想法。

% それがわかるからこそ、直哉は肩をすくめるしかない。\\
正是因为知道了这一点,直哉只得耸了耸肩。\\

% 「なんだよ、朝一緒に登校しようって言ったのは白金さんだろ」
「什么啊,不是白金同学说想要早上一起去学校的吗」

% 「ええ、そうよ。その通りだわ」\\
「是是,是的啊。就是这样啊」\\

% 小雪はムスッとしたまま、首を縦に振る。\\
小雪不爽地点着头。\\

% 「昨日は笹原くんにペースを乱されちゃったけど、今日からは私の反撃が始まるんだから。そのためにも、些細なことだろうとあなたにアドバンテージなんてあげたくないの。だから早く来たっていうのに……もういるし」
「虽然昨天被笹原君给打乱了节奏,但是从今天开始我就要反击了。为此,就算是再琐碎的事我也不会给你领先的机会了。所以明明想早一点到的……你却已经到了。」

% 「待ち合わせ場所にどっちが先に着くかなんて些細な問題じゃね?」
「谁先到约定的地方难道不是琐碎的问题吗?」

% 「だったらなんであなたも早く来ちゃったのよ!」
「那你为什么要先到啊!」

% 「いやあ、ちょっと恥ずかしいんだけどさ……」\\
「怎么说呢,理由稍微有点羞耻啊……」\\

% 直哉は頬をかきつつ、素直に打ち明ける。\\
直哉挠了挠脸颊,老实地坦白道。\\

% 「朝から白金さんと会えると思ったら目が冴えて。ついつい早起きしちゃったんだよ」
「早上想着能够见到白金同学了,就有些兴奋地睡不着了。一不留神就起了个大早。」

% 「……私に会えるから?」
「……因为能见到我?」

% 「そうそう。遠足前の子供みたいだろ?」
「对对。就像远足前的小孩一样对吧?」

% 「ふ、ふーん、そう。たしかにお子様ね。ふーーーん」\\
「嗯,嗯~,这样啊。确实像小孩子一样呢。哼——。」\\

% 小雪はつんと澄ました顔でうなずいてみせる。
小雪用她那张白净的脸若无其事地点了点头。

% ただし口元はニヤついていて、そわそわしているのが丸わかりだ。
只不过她的嘴角带着微笑,看得出她有些坐立不安。

% わりともう、読み取る必要がないくらいにはいろんなものがダダ漏れである。\\
甚至连读心的必要都没有,就什么都表现了出来。\\

% (うーん。それを指摘すると絶対怒るよな。ここは黙ってよ)\\
(嗯~。要是指出这一点的话绝对会生气的吧。现在还是保持沉默好了。)\\

% 直哉もそろそろ学習したので、余計なことは言わずにいた。
直哉也差不多成长了一些,并没有说些多余的话。

% 昨日もいろいろと分かりやすかったが、今日はもっと分かりやすい。喫茶店から出たあとのことを直哉はぼんやりと思い起こす。\\
昨天就已经很好懂了,但今天却更好懂。直哉隐约地想起了昨天从咖啡厅出来后的事。\\

% あのときはちょうど、外に出ると日が沈みかけていた。茜色に染まる空のもと、街並みは買い物帰りの主婦や学生たちでにぎわいをみせる。
那个时候正好,出来的时候太阳已经西沉。在被染成暗红色的天空下,购物归来的主妇和学生们挤满了街道,显得十分热闹。

% そんな景色の中に小雪が「それじゃまたね」と消えかけたので、直哉は慌ててそれを呼び止めた。\\
「那明天见了」这样说着,小雪消失在了那样的景色中。直哉慌慌张张地叫住了她。\\

% 『ちょっと待って。そういや白金さん、家はどの辺?』
『稍微等一下。说起来白金同学,你家住哪边?』

% 『四ツ森の方だけど……それがなにか?』
『在藩境塚那边……怎么了?』

% 『あー、俺とは逆方向か。いや、遅くなったし送っていこうかと』
『啊~,和我是相反的方向啊。不是,我想着有些晚了,还是送送你比较好。』

% 『けっこうよ。ただの同級生にそこまでしてもらう義理はないわ』
『不用了。只不过是同级生而已,你没有义务做到这种程度。』

% 『いやでも、そろそろ暗くなるしさ。好きな子の心配をするのは男として当然じゃないか?』
『但是,已经慢慢暗下来了。作为一个男人担心一下喜欢的人是天经地义的吧?』

% 『しゅ、き…………!?』\\
『稀,饭…………!?』\\

% 小雪の顔が音を立てて真っ赤に染まる。
话一出口,小雪的脸便染上了一抹红晕。

% しかしそのまま何度も深呼吸してみせてから、ふんぞりかえってみせるのだ。\\
但是她就这样几次深呼吸之后,傲慢地回答道。\\

% 『ふ、ふん。そんな浮ついたことが言えるのは今日までよ。見てなさい……明日から私の逆襲が始まるの。グイグイとアタックして、私なしでは生きていけないくらいに骨抜きにしてやるんだから!』
『哼,哼——。你也就只有现在能说出这种轻浮的话了。看着吧……从明天开始我就要逆袭了。开始不断地向你进攻,让你没有我就活不下去的程度地迷上我。』

% 『いやでも、もう十分好きなんだけど? これ以上骨抜きにされたら俺、クラゲになるしかないんだけど』
『但是,我已经足够喜欢你了哦?再喜欢下去的话,怕是连骨头都不会剩下了』

% 『そういうのじゃなくて、私に逆らえないようにしたいの!』\\
『我不是这个意思,我是想让你变得无法违逆我而已!』\\

% 小雪はぷりぷりと湯気を立てて怒る始末。
小雪气呼呼地开始发火。

% 言ってることは暴君めいているが、いっぱいいっぱいな様子もあって一切怖くはなかった。
虽然说的话很像暴君,但那气冲冲的样子却一点也不恐怖。

% そんな彼女を直哉は冷静に分析する。\\
对着那样的她,直哉冷静地分析到。\\

% (うーん……なるほど。プライドと好意、あとちょっと好奇心ってところか)\\
(嗯~……原来如此。自尊心和好感,然后还带有些好奇心吗。)\\

% 直哉のことは好きだが、素直に言うのはプライドが許せない。おまけに先ほどから直哉のペースにハマっていて完全に面白くない。
虽然是喜欢直哉的,但是自尊心不允许自己就这样坦率地说出来。再加上刚刚开始就一直被直哉牵着鼻子走,一点也不有趣。

% だから何とか彼より優位に立つため、『籠絡』という手を選んだのだろう。
所以为了比他更有优势,才选择了『笼络』的手段吧。

% 直哉を骨抜きにすれば、多少は自尊心が満たされる。こんなふうに直哉からグイグイ来られて戸惑うこともなくなる。
要是让直哉迷上了的话,多少能满足一下自己的自尊心。就不会像这样被他不断地进攻搞得很困扰了吧。

% ……おおよそ、そんな腹づもりだろう。\\
……大概,就是这样的计划吧。\\

% とはいえそれは直哉にとっては好都合。好きな子がグイグイとアタックしてくれるのだ。完全にご褒美である。
虽然这样说,但这对直哉来说也很好吧。喜欢的人不断地进攻上来。完全就是对他的奖励。

% それがどうも小雪にはわからないらしい。自分の目的優先で、手段が完全に破綻している。
似乎小雪还没搞清楚。以自己的目的优先,这样的手段满是破绽。

% あまりにも子供じみているというか、ポンコツすぎるというか……。\\
不知道说她有些幼稚好,还是有些自大好……。\\

% 『白金さんはかわいいなあ』
『白金同学真可爱啊』

% 『……なんかそのニュアンス、心底腹立つんだけど?』\\
『……总觉得你的言下之意让人有些打心底里生气?』\\

% しみじみとこぼした直哉のことを、小雪はにらむ。野生の勘でイラっとしたらしい。
小雪对着感慨万千的直哉怒目而视。由于她野性的直觉生气了起来的样子。
