% Auto-slash processed
% \subsection{察しのいい理由}
\subsection{观察力的缘由}

% 大事なことを言い終えたからか、小雪はほっとしたようだった。ようやくドーナツに手を伸ばし、ちまちまとかじり始める。\\
或许是因为说完了最重要的事情,小雪好像松了口气。她终于伸手拿起甜甜圈,小口地咬起来。\\

% 「なんだか調子が狂っちゃうわ……あなたって変な人よね」
「你可真是个奇怪的人……弄得我都不对劲了。」

% 「ああうん、よく言われる」
「嗯,经常有人这么说我。」

% 「でしょうね」\\
「我就知道。」\\

% 小雪は呆れたように言ってのける。形のいい唇をつり上げて、小馬鹿にするような笑みをうかべてみせた。\\
小雪半是惊讶半是无奈地说完,翘起线条优美的嘴唇,促狭地对他笑了起来。\\

% 「あなたみたいな変人が私とお茶できるなんて、本当ならありえないことなんだから。光栄に思ってちょうだいな」
「你这样奇怪的人,本来是根本不可能和我一起喝下午茶的。好好地记住这种光荣吧。」

% 「いやあ、ほんと光栄だよ。『笹原くんとお茶できてうれしい』なんて言ってもらえて」
「啊,真的是很光荣,能让你觉得『可以和笹原君一起喝茶好开心』。」

% 「言ってない! 断じてそんなこと言ってないわよ!?」\\
「我没有那么说! 我绝对没有说过那种话哦!?」\\

% 必死になって否定するも、耳まで真っ赤な反応でバレバレである。
就算拼命否定,小雪的脸还是红到了耳朵根,可谓是暴露无遗。

% 小雪はひとしきり叫んでから、ほかの客の注目を集めてしまったことに気付いたのだろう。途端にしゅんっと小さくなって、コーヒーをすする直哉に恨みがましい目を向ける。\\
她尖叫过后大概是发现自己吸引了其他客人的注目,身子立即好像缩小了一圈。接着,小雪朝小口啜饮咖啡的直哉投去怨恨的视线。\\

% 「ほんっと、どんな耳してるのよ……そんなこと私、一言も言ってないのに……」
「你到底,到底是长着什么样的耳朵啊……那种话,我明明一个字都还没说出来……」

% 「いやだって、白金さんの本心くらい簡単にわかるしな」\\
「毕竟,白金同学心里在想什么,我一眼就能看出来啊。」\\

% 直哉は平然と答えてみせる。
直哉淡然地回答道。

% 小雪の真意を測るのはそう難しいことではない。単語のアクセントや目線の動き、髪をかきあげる身振りなど。そうしたものをしっかり注意していれば誰でもわかる。\\
推测小雪心中的想法并不是什么难事。一句话中语气所强调的部分,视线的动向,拨撩头发的举止等等,只要留心观察这些,谁都能明白她的心思。\\

% 「本当に……? 胡散臭いわね」\\
「真的吗……?听起来好可疑。」\\

% 小雪はじとーっとした目で直哉を見つめてから、ふといたずらっぽい笑みを浮かべてみせる。財布から百円玉を取り出して、両手をグーにして直哉に突きつけた。\\
小雪直勾勾地盯着直哉看了一会儿,忽然噗哧地笑起来,然后从钱包里拿出一枚100日元硬币,两手攥紧伸给直哉看。\\

% 「それじゃ問題よ。コインはどっちの手の中にある?」
「那我考你一下。硬币在哪只手里面?」

% 「膝の上だろ」
「在你的腿上吧。」

% 「…………正解」\\
「…………答对了。」\\

% 小雪はげんなりした顔で両手を開いてみせる。
小雪无奈地展开双手。

% はたしてそこには何もなかった。膝の上からコインを摘まみ上げて、小雪は信じられないものを見るような目で直哉を見つめる。\\
两只手里都是空空的。她从腿上拿起硬币,用难以置信的眼神看着直哉。\\

% 「ほんとに勘がいいのね……そういえば、昨日もモデルのスカウトっていうのが嘘だって見抜いていたけど。あなた、探偵かなにか?」
「你的眼力真的很好呢……说起来,昨天你也看穿了那个假装是星探的人。你是侦探什么的吗?」

% 「高校生探偵なんてアニメやゲームだけの存在だろ。俺はふつうの高校生だって」
「高中生侦探这种东西只可能在动画片或者游戏里才有吧。我只是个普通的高中生。」

% 「ふつうの高校生はこんな手品できないわよ」\\
「普通的高中生才不会这种魔术呢。」\\

% 小雪はなおも疑わしげな目を向けてくる。
小雪依旧是一副充满怀疑的视线。

% 恩人に対する態度ではないが、直哉は気にせず肩をすくめるだけだ。\\
虽然这不该是对待恩人的态度,但直哉没有在意,只是耸了耸肩。\\

% 「ま、よく聞かれるよ。おまえのそのスキルはなんだ、ってな」
「经常有人这样问我的,『你从哪里得来这样的技能啊』之类。」

% 「そりゃ誰でも気になるわよ。いったいどんな特訓を積んだわけ?」
「这个当然每个人都会在意啊,所以你到底是经过了怎么样的特训?」

% 「そんな大袈裟なもんじゃないさ」\\
「其实根本没有那么夸张的。」\\

% 直哉は軽く苦笑する。
直哉轻轻苦笑起来。

% 特に隠すほどのことではない。ただちょっと、そういうスキルを会得する必要に迫られただけだ。\\
反正也没有什么好隐瞒的。只不过,当时他是迫于一点客观需要,才学会了那样的技能。\\

% 「実は……俺が小さい頃、母親が大病を患ってさ。一時期ほとんど寝たきりだったんだ」
「其实……我小的时候,母亲得过一场大病。有一段时间几乎是昏迷的。」

% 「……えっ」\\
「……啊。」\\

% その切り出しが予想外だったのか、小雪は小さく息を飲んだ。
小雪轻轻吞下一口气,她似乎没有想到直哉会这样开口。

% おかまいなしで直哉は続ける。\\
直哉则继续不在意地讲述起来。\\

% あれは直哉が六つの頃だ。
那是直哉六岁的时候。

% 母親がある日突然病に倒れ、緊急入院した。
有一天,母亲突然病倒,紧急入院了。

% そのままほとんど一日中寝たきりとなり、人工呼吸器や幾多の管で繋がれ、意思の疎通ができない状態が続いた。\\
那之后她便整日昏睡,靠着人工呼吸机和众多管线维持生命,很长一段时间内连表达意思都做不到。\\

% それでも直哉は毎日病室に通い、母親の看病に勤しんだ。必死になって母親の表情を注目し、なにを求めているのか読み解こうとした。\\
即便如此,直哉依旧每天去病房里照看母亲。他拼命地观察母亲的表情,试图理解母亲的需要。\\

% 目線だけで欲しいものを読み取ったり。
直哉能从视线判断母亲需要什么物品。

% みじろぐ頻度で苦痛を察したり。\\
能从身体动作的频度判断她的痛苦。\\

% 「まあ、ガキにできることなんて限られてたけどさ。そんなのを繰り返すうちに、自然と人の言いたいことがわかるようになったんだよな」
「虽然当时我还是个小不点,能做的事情很有限就是了。不过就那样一直继续下去,自然而然地,我就知道了别人心里想说的东西。」

% 「そう……お母様のために……」\\
「这样啊……你是为了母亲才……」\\

% 小雪は口元を押さえて目を丸くする。
小雪捂住嘴,瞪圆了眼睛。

% そうして恐る恐ると問いかけることには――。\\
然后,小心翼翼地接着问道——。\\

% 「その……お母様は、今どうしてるの……?」
「那……你的母亲,现在怎么样了……?」

% 「…………遠いところにいるよ」
「…………她在很远很远的地方。」

% 「っ……!」\\
「……!」\\

% 小雪の顔がさっと青ざめる。
小雪的脸顿时变白了。

% 一方、直哉は平然と続けた。\\
直哉则依旧淡定地说道。\\

% 「たぶん今はカリブ海あたりじゃないかな」
「大概,现在是在加勒比海那一带吧。」

% 「……は?」
「……啥?」\\

% 「いや、あれから親父の海外出張について行ったんだよ」\\
「啊,那之后我老爸去海外出差,于是她也就跟着去了。」\\

% 一時は余命宣告を受けた母親だが、奇跡の回復を遂げ、病に倒れる前以上に元気になった。
一时间被下了病危通知书的母亲,后来居然奇迹般地康复,并且变得比病倒之前更健康。

% おかげで今は夫婦揃っての海外を満喫している。直哉も高校生になったため、置いていっても問題ないと判断したらしい。
因此才得以和丈夫一起在海外享受夫妇生活。他们大概是判断直哉已经长成了高中生,一个人生活也没什么问题。

% 毎月メールで元気な知らせが届くのだが、そこに添付されている写真がどれも夫婦ラブラブで、息子としては安心していいやら複雑に思うべきなのか困ってしまうのが常だった。\\
虽然每月他们都会寄信给直哉表示自己过得很好,但信中附上的照片总是浸透了恩爱气氛。作为儿子,直哉也很犹豫,不知道自己是该安心还是该感觉心情复杂。\\

% そう告げると、小雪は悔しそうにドーナツをかみしめる。\\
听到这里,小雪一副不甘心的模样在甜甜圈上咬了一口。\\

% 「ただただ紛らわしい……!」
「这个结局太有欺骗性了……!」

% 「あはは、それもよく言われる」\\
「哈哈,经常有人这么说呢。」\\

% いわゆる鉄板ネタというやつである。
正是所谓的标准结局。
