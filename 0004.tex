% \subsection{察しのいい理由}
\subsection{察しのいい理由}

% 大事なことを言い終えたからか、小雪はほっとしたようだった。ようやくドーナツに手を伸ばし、ちまちまとかじり始める。\\


% 「なんだか調子が狂っちゃうわ……あなたって変な人よね」


% 「ああうん、よく言われる」


% 「でしょうね」\\


% 小雪は呆れたように言ってのける。形のいい唇をつり上げて、小馬鹿にするような笑みをうかべてみせた。\\


% 「あなたみたいな変人が私とお茶できるなんて、本当ならありえないことなんだから。光栄に思ってちょうだいな」


% 「いやあ、ほんと光栄だよ。『笹原くんとお茶できてうれしい』なんて言ってもらえて」


% 「言ってない! 断じてそんなこと言ってないわよ!?」\\


% 必死になって否定するも、耳まで真っ赤な反応でバレバレである。


% 小雪はひとしきり叫んでから、ほかの客の注目を集めてしまったことに気付いたのだろう。途端にしゅんっと小さくなって、コーヒーをすする直哉に恨みがましい目を向ける。\\


% 「ほんっと、どんな耳してるのよ……そんなこと私、一言も言ってないのに……」


% 「いやだって、白金さんの本心くらい簡単にわかるしな」\\


% 直哉は平然と答えてみせる。


% 小雪の真意を測るのはそう難しいことではない。単語のアクセントや目線の動き、髪をかきあげる身振りなど。そうしたものをしっかり注意していれば誰でもわかる。\\


% 「本当に……? 胡散臭いわね」\\


% 小雪はじとーっとした目で直哉を見つめてから、ふといたずらっぽい笑みを浮かべてみせる。財布から百円玉を取り出して、両手をグーにして直哉に突きつけた。\\


% 「それじゃ問題よ。コインはどっちの手の中にある?」


% 「膝の上だろ」


% 「…………正解」\\


% 小雪はげんなりした顔で両手を開いてみせる。


% はたしてそこには何もなかった。膝の上からコインを摘まみ上げて、小雪は信じられないものを見るような目で直哉を見つめる。\\


% 「ほんとに勘がいいのね……そういえば、昨日もモデルのスカウトっていうのが嘘だって見抜いていたけど。あなた、探偵かなにか?」


% 「高校生探偵なんてアニメやゲームだけの存在だろ。俺はふつうの高校生だって」


% 「ふつうの高校生はこんな手品できないわよ」\\


% 小雪はなおも疑わしげな目を向けてくる。


% 恩人に対する態度ではないが、直哉は気にせず肩をすくめるだけだ。\\


% 「ま、よく聞かれるよ。おまえのそのスキルはなんだ、ってな」


% 「そりゃ誰でも気になるわよ。いったいどんな特訓を積んだわけ?」


% 「そんな大袈裟なもんじゃないさ」\\


% 直哉は軽く苦笑する。


% 特に隠すほどのことではない。ただちょっと、そういうスキルを会得する必要に迫られただけだ。\\


% 「実は……俺が小さい頃、母親が大病を患ってさ。一時期ほとんど寝たきりだったんだ」


% 「……えっ」\\


% その切り出しが予想外だったのか、小雪は小さく息を飲んだ。


% おかまいなしで直哉は続ける。\\


% あれは直哉が六つの頃だ。


% 母親がある日突然病に倒れ、緊急入院した。


% そのままほとんど一日中寝たきりとなり、人工呼吸器や幾多の管で繋がれ、意思の疎通ができない状態が続いた。\\


% それでも直哉は毎日病室に通い、母親の看病に勤しんだ。必死になって母親の表情を注目し、なにを求めているのか読み解こうとした。\\


% 目線だけで欲しいものを読み取ったり。


% みじろぐ頻度で苦痛を察したり。\\


% 「まあ、ガキにできることなんて限られてたけどさ。そんなのを繰り返すうちに、自然と人の言いたいことがわかるようになったんだよな」


% 「そう……お母様のために……」\\


% 小雪は口元を押さえて目を丸くする。


% そうして恐る恐ると問いかけることには――。\\


% 「その……お母様は、今どうしてるの……?」


% 「…………遠いところにいるよ」


% 「っ……!」\\


% 小雪の顔がさっと青ざめる。


% 一方、直哉は平然と続けた。\\


% 「たぶん今はカリブ海あたりじゃないかな」


% 「……は?」


% 「いや、あれから親父の海外出張について行ったんだよ」\\


% 一時は余命宣告を受けた母親だが、奇跡の回復を遂げ、病に倒れる前以上に元気になった。


% おかげで今は夫婦揃っての海外を満喫している。直哉も高校生になったため、置いていっても問題ないと判断したらしい。


% 毎月メールで元気な知らせが届くのだが、そこに添付されている写真がどれも夫婦ラブラブで、息子としては安心していいやら複雑に思うべきなのか困ってしまうのが常だった。\\


% そう告げると、小雪は悔しそうにドーナツをかみしめる。\\


% 「ただただ紛らわしい……!」


% 「あはは、それもよく言われる」\\


% いわゆる鉄板ネタというやつである。

