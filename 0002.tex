% Auto-slash processed
% 毒舌にもかまわずグイグイいく
\subsection{无惧毒舌不断攻略}

%  次の日の昼休み。
第二天午休时间。

%  学校の廊下を歩いていると、予期せぬ人物が直哉の前に立ちはだかった。
直哉穿过学校的走廊时,有个意想不到的人拦在了直哉面前。\\

% 「あなたが『笹原直哉くん』よね。昨日はどうも」
「你就是『笹原直哉君』吧。昨天的事谢谢你了」\\

%  腰まで届く髪は銀。宝石のようにきらめく瞳は海の色。
%  その面立ちは整いすぎていて、よくできたCGと言われても納得するほどだ。肌は透けるように白く、小さな唇からこぼれ落ちる声も鈴を転がすように澄んでいる。\\
一头及腰长发是银色,如宝石般闪烁的眼眸则是海之色。

面孔精致到了极端的程度,说是工艺高超的CG也不为过;肌肤雪白通透,小小嘴唇中发出的声音也像银铃一样清亮。\\

%  ただし、こちらに向ける視線はやけに鋭い。
只是,投向直哉的眼神却过分锐利。

%  小柄なその体からは殺気がほとばしっており、腕を組んで仁王立ちするその姿は、まさに金剛力士像だ。\\
释放杀气的娇小身躯,以及双手环抱两脚分开的站姿,就如同金刚力士一样。\\

%  おかげで廊下に溢れるほかの生徒たちはざわざわする。
多亏了她的这副模样,挤满了走廊的其他学生们开始七嘴八舌地议论起来。

%  一緒にいたチャラめの男子生徒、|河野巽《こうのたつみ》も目を丸くして直哉に耳打ちした。
直哉身旁的轻浮男生河野巽睁大眼睛,对他耳语道。\\

% 「おいおい直哉、おまえ……『猛毒の白雪姫』と何かあったのかよ?」
「喂,我说直哉你啊……是和『剧毒的白雪公主』发生了什么吗?」

% 「ああ、うん。昨日ちょっとな」
「啊,嗯,昨天确实发生了点事情」\\

%  直哉は鷹揚にうなずいてみせる。
直哉大大方方地点头。

%  あのときは顔を見なかったが……長い銀髪で、おそらく彼女だろうとは察しがついていた。
当时虽然没看到对方的脸……不过,从这头银色长发推测,直哉觉得就是这个女生没错了。\\

% (ただ、こうやって再エンカウントするとは思わなかったよなあ)
(不过,真没想到居然会是这种形式的再会。)\\

%  彼女の名前は|白金小雪《しろがねこゆき》。
%  直哉と同じく、大月学園の二年生だ。
%  容姿端麗かつ頭脳明晰、おまけにスポーツ万能というその評判は、別のクラスである直哉の耳にも届いていた。
她叫白金小雪。

和直哉一样是大月学园的二年级学生。

别人都说她不仅容貌端正、头脑清晰,而且还精通各种体育运动,就连不同班的直哉也知道别人对她的这些评价。

%  ただし、どちらかというと広まっているのは賞賛だけではなく、悪名の方が多かった。
不过要说的话,传开的其实不只有赞赏,反倒是恶名更多一些。

%  ぼんやりする直哉を前にして、小雪は淡々と告げる。\\
面对着愣住的直哉,小雪淡淡地开口。\\

% 「昨日はどうもありがとう。お礼を言いそびれちゃったからわざわざ探したのよ。制服だったから、同じ学校だってことは分かったし」
「昨天真是多谢了。我错过了道谢的机会,所以才特地来找你的。因为我看到了你的校服,知道是同一个学校的学生」

% 「別にお礼なんていいのに」
「不用那么客气啦」

% 「そういうわけにはいかないわ」
「那怎么行」\\

%  小雪はじろりと直哉をねめつけて、長い髪をかきあげてみせる。
小雪直勾勾地瞪着直哉,然后撩起长发。\\

% 「あの程度のことで、恩を売ったなんて思われちゃ困るもの。そうでなきゃ、この私がただの男子生徒Aにわざわざ声なんかかけるわけないでしょ」
「我可不想让你觉得那种小事就算是卖了我一个人情。不然,我也不可能特地找一个普通男生A搭话啊」

% 「はあ」
「哈啊」\\

%  完全無欠の美少女、白金小雪の唯一の欠点。
%  それがこの毒舌だ。
完美无缺的美少女白金小雪唯一的缺点。

那就是毒舌。

%  入学から一年あまりが経過して、これまで何人もの男子生徒がアタックし、その全員が彼女の苛烈極まりない口撃によってノックアウトさせられたという。
%  結果、ついたあだ名が『猛毒の白雪姫』。
小雪入学到现在过了一年多,这段时间她曾遇到过多名男生对她发动攻势,然而所有人都在她尖锐至极的言语下被踢出局了。

结果就诞生了这样的称号『剧毒的白雪公主』。\\

% 「今日もキッツイなあ、猛毒の白雪姫……」
「今天也不留情面啊,剧毒的白雪公主……」

% 「ねえ……何があったか知らないけど、言い方ってものがあるよね」
「那个……虽然不知道发生了什么,但应该有更好的说法吧……」\\

%  ギャラリーたちも眉をひそめ、ひそひそと言葉を交わし合う。
%  だがしかし、小雪は目つきをさらに鋭くして続ける。
围观群众皱起眉头窃窃私语,小雪的眼神却愈发锐利。\\

% 「昨日はたしかに少し怖かったけど……あなたが手を出さなくても、私ひとりでどうにかなったんだから。くれぐれも、白馬の王子さま気取りはやめてちょうだいね。私、借りを作るのは好きじゃないの」
「昨天我确实有一点害怕……但是就算不用你干涉,我一个人也能想办法解决的。拜托你务必不要觉得自己变成了白马王子。我也不喜欢欠别人人情债」

% 「おお、わかった」
「哦,我明白了」\\

%  直哉はあっさりとうなずく。
%  彼女の言いたいことは、よーーーく理解した。
直哉爽快地点头。

他完~~~全地明白了小雪的意思。\\

% 「つまり白金さんは俺にお礼がしたいから、放課後どこかに誘いたいんだな?」
「也就是说,白金同学希望对我道谢,所以想要放学后约我到哪里去吧?」

% 「…………は?」
「…………啥?」

% 『…………はあ?』
『…………哈?』\\

%  小雪だけでなく、周囲の生徒たちも目を丸くした。
%  おおむね『何を言ってるんだこいつは』という反応だ。
不只是小雪,周围的学生也都瞪大了眼睛。

所有人的反应大致都是:『这家伙到底在说什么?』

%  だがしかし、すぐに小雪の様子がおかしくなる。一瞬で耳の先まで真っ赤に染まり、裏返った声を上げた。
不过,小雪的模样很快就出现了异变。她一瞬间脸红到了耳根,发出尖叫。\\

% 「い……いったい何を言ってるのよ!? 今の私のセリフを聞いて、どうしてそんな結論になるわけ!?」
「你……你到底在说什么!?你有听我说吗!?为什么会得到那种结论!?」

% 「いやだって、わかりやすいから」
「不,因为你说得很好懂啊」\\

%  直哉はあっさり告げるしかない。
直哉只得坦率地解释道。\\

% 「『怖かった』ってのは本当だろ。あとはほとんど強がりだ」
「『害怕』是真的吧,后面基本都在逞强」

% 「っ……!」
「……!」

% 「で、『借りを作りたくない』ってのも本当だけど、ちょっとニュアンスが違う。本音は『お礼がしたい』だ」\\
「然后『不想欠人情债』也是真的,不过话外音有点不一样,其实你是『想回礼』」\\

% そして、昼休みもそろそろ終わる。
% 小雪が本当にお礼がしたいと思っているのなら、放課後しか時間はないだろう。
考虑到午休快要结束,如果小雪是真的想要回礼,那就只有放学后才可以。

% これくらい、小雪の口ぶりや態度、状況などを鑑みれば誰にでもわかることだった。
这些信息,只要从小雪的口吻,态度和状况来推断,任何人都能看得出来。

% ぽかんと言葉を失う小雪に、直哉はつらつらと畳み掛ける。\\
向着呆呆说不出话的小雪,直哉认真地继续说道。\\

% 「今日はちょうどバイトがないんだ。部活もやってないし、放課後は空いてる。白金さん、どうする?」
「今天正好没有兼职。我又不参加社团,放学后是有空的。怎么样,白金同学?」

% 「だ、だから、私は……うっ、ううう……!」\\
「所、所以,我……唔、唔唔唔……!」\\

% 小雪はぷるぷると震え、俯いてしまう。
小雪低着头,不停地颤抖着。

% そのまましばらく待ってみれば……彼女はぼそぼそと小声でこぼす。\\
直哉等了她一会儿……她才小声嘀咕着说。\\

% 「あの、もしよかったら…………で、待ってるから……だから、その……」
「那个,你方便的话,…………等你……所以,那个……」

% 「うん。わかった、放課後に正門前で待ち合わせだな。了解」
「嗯,知道了。放学后在正门前见面对吧,了解」

% 「なんでちゃんと聞こえるのよ!? そこは普通、聞こえなくて聞き返すってのがセオリーじゃなくって!?」
「为什么你听得那么清楚啊!?这种时候听不清楚然后向对方再问一遍才是常理好不好!?」

% 「いや俺、生まれてこのかた聴力検査で引っかかったことないからさ」
「不是,我从出生到现在,从来没在听力检查上遇到过问题啊」

% 「ぐううっ……! こ、この……!」
「唔……!你、你这……!」

% 「この?」
「嗯?」

% 「笹原くんの……健康優良児ぃぃぃぃ!」\\
「笹原君你个……三好学生——!」\\

% 褒めているとしか思えない捨て台詞を残し、小雪は真っ赤な顔のまま逃げ出してしまった。\\
小雪抛下一句怎么想都是夸奖的话,然后满脸通红地逃走了。\\

% 「……白金さんって、案外……」
「……没想到白金同学居然……」

% 「ねえ……」
「是啊……」

% 「可愛いところもあるんだねえ……」\\
「也有可爱的地方啊……」\\

% 彼女が消えた方へと、ギャラリーたちは生温かい目を向ける。
围观群众向她逃走的方向投以温暖的视线。

% そんななか、友人の河野がぽんっと直哉の肩を叩いた。
这时,直哉的朋友河野拍了拍他的肩膀。

% 呆れたような、笑いを堪えるような顔で言うことには――。\\
他一副傻眼的模样,又好像是憋着笑——。\\

% 「おまえのその読心スキル、今日も絶好調だな」
「你的读心术,今天也是状态绝佳啊」

% 「これくらいみんなわかるだろ?」\\
「这种东西大家都应该能看出来才对吧?」\\

% 直哉は不思議そうに首をかしげるだけだった。\\
——而直哉却只是感到疑惑不解。\\

\vspace{2\baselineskip}

% これはやたらと察しのいい少年が、毒舌クーデレ少女に完勝を続ける物語。
这是一个观察力太好的少年,不断完胜毒舌冷娇少女的故事。
